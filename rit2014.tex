  \item сочи2014 с точки зрения it: с нуля, непонятно на чём, высокая нагрузка, несдвигаемые сроки, высокий уровень угроз ИБ
  \item если техническую платформу выбирают менеджеры, то они выбирают bitrix, беда-беда (и не думают о технологиях вообще)
  \item технологический тендер сочи2014 выйграл Microsoft с Azure, вот так вот
  \item сочи2014 194тыс посетителей - пик, 55 млн уникальный посетителей (Google Analytics)
  \item предыдущие цифры без учёта Китая, где с осени 2012 google analytics не работает
  \item сочи2014: 100\% uptime, 4 мегабайта на главную и около 4 секунд на страницу
  \item email-рассылки от Badoo...
  \item продуктовых готовых рецепетов про email рассылки в общем-то нет, каждый проект уникален
  \item email-рассылка: нужно понимать нагрузку, какие ip-адреса и домены используются, не забудьте про: DKIM, SPF, PTR, DMARC!
  \item email-рассылка: самим, если большая техническая команда, полный контроль, и тд; готовое - если интеграция с CRM и/или малый бизнес
  \item грубая оценка, 5 миллионов в день с одного сервера; имейте второй сервер если у вас доход от потери писем больше \$2000
  \item хороший сервис рассылок обязан: доставлять за минуты, не должно быть проблем с доставкой, работает: bounce, FBL, ...
  \item статистика обязательна(sic!): процент открытий, кликов; размер очереди отправки; скорость доставки; число ошибок отправки
  \item микрофреймворк для писем - the must: подстановка параметров авторизации, проверка получателя --- позволит просто расширять письма
  \item все письма в одином конфиге! ссылка на описание; приоритет письма; тип уведомлений; контакт "автора" письма
  \item асинхронная генерация писем: изоляции и производительность - нужно разделить событие, генерация и отправку писем на различные этапы
  \item мониторить аномалии - обязательно, большинство значительных колебаний - это почти всегда баги
  \item ведите историю доставляемости: когда (с-по), почтовик (блеклист), причина, как решали ...
  \item как починить письмо в спаме? - написать в почтовый сервис!, указать: IP, домен, код ошибки, время, и тд
  \item мониторинг email-рассылки: seed-мониториг, история подтверждений, среднее время доставки
  \item FYI: http://postmaster.live.com , http://postmaster.mail.ru , http://postoffice.yandex.ru 
  \item да, кстати. люди не любят письма! с приглашениями нужно быть предельно осторожными
  \item популярные ошибки email-рассылок: резкое изменение объёма трафика, мгновенный переход на новый IP и домены, большие компании, ...
  \item заметки практекующих e-mail маркетологов, стратоплан
  \item как умным людям работать с другими умными людьми?! это между прочим сложно ...
  \item почему email-маркетинг?! мы не умеем пользоваться другими каналами; email обычно есть, it-шники читают письма каждый день, 1:1
  \item habrahabr даёт не плохой поток подписчиков-it-шников, если писать хорошо!
  \item после набора абоненской базы нужно сначала настроить доверительные отношение, иначе bullshit detector, сведёт эффект на ноль!
  \item психологи говорят: о хорошем человек говорит 5-7 знакомым, а о плохом 10-12
  \item email-маркетинг, конкретная полезная метрика - Open Rate писем; процент отписок; количество повторных продаж
  \item .oO андрогогика --- теория обучения взрослых людей
  \item pulsedb - база данных для хранения временных рядов
  \item проблема: хранение и обработка большого количества временных статистик, +графики
  \item pulsedb vs rrdtool, graphite, opentsdb, influxdb, ...
  \item админы любят то, что на C; программисты наоборот, любят то, что не на C
  \item о rrdtool: медленный (fork на замер), нельзя склеить метрики, примитивный
  \item о graphite: сложно склеивать метрики на лету, графики стороит сам
  \item о opentsdb: hadoop, рисует графики, склеивает ряды, нет realtime-а, огромный overhead на данные
  \item о influxdb: go, развитие решение, SQL-like, агрегация рядов, умеет события
  \item pulsedb - библиотека/демон, замер: UTC+имя+теги, компактное неточное хранение, websocket-подписка
  \item 14bit на замер хватит всем! (c) Макс Лапшин
  \item pulsedb: HTTP Upgrade + текстовый протокол, собственный формат хренения, последняя минута в памяти
  \item pulsedb: 2 байта на замер, 14 бит на значение, 2 бита на порядок; 25 GB в месяц (10 млрд замеров)
  \item pulsedb транспорт по сети: statefull протокол, неточные данные
  \item pulsedb умеет делать внешнюю авторизацию, внешний resolve-инг графиков
  \item pulsedb планы: MySQL интерфейс, GUI для данных
  \item schema-less в PostgreSQL9.4, Бартунов, нашумевший доклад вживую
  \item сервисная архитектура изменила подход к СУБД, ACID -> BASE, масштабировние и производительность
  \item hstore в postgresql с 2003-го года, Key-Value, бинарное хранилище
  \item json в postgresql - только строка as is, это медленно и не эффективно
  \item а вот jsonb - это быстро и эффективно, и в самом ядре; a hstore и json - legacy
  \item text vs json vs jsonb: хранении совпадает, вставка незначительно медленее, а выборка намного быстрее
  \item jsonb в postrgesql в перфоманс тестах впечатляет! в зависимости от case-а частенькое эффетивнее mongodb
  \item "хочется купить что-небудь красненькое" - women oriented query: create index ... using vodka
  \item граница между rational и nosql решениями стирается
  \item postgesql9.4: jsonb, dynamic background workers, in-memory columnar storage, logical replication
  \item распределённые отказоустойчивые fs: glusterfs, ceph, parallels cloud storage
  \item wtf?!: объём, скорость, доступность, стоимость
  \item glusterfs: posix-compatible, no metadata server, heavy portable, nfs/cifs/native
  \item ceph --- стабильность под большим вопросом, неочевидный процесс востановления
  \item pcs (parallels cloud storage) в первую очередь ориентирован на хранение образа VM
  \item анатомия web-сервиса: ...
  \item backend: принять соединение, аутентификация, авторизация, сессия...
  \item ..., распарсить url, rate limiting, бизнес-логика, кэширование, формирование ответа, шаблона, всё ...
  \item неблокирующий ввод/ввывод: select => ready to read/write, do read/write unltil EAGAIN
  \item многозадачтоность (мнопользовательность): процессы, нити(threads) на уровне ОС, event-loop, кооперативная многозадачность, реактор
  \item gevent - с точки зрения кода выглядит хорошо, и работает неплохо - переключение происходит именно тогда, когда надо
  \item ещё одна работа web-сервиса: общение с "db": request/response vs pipeling; N: одно vs pool, etc ...
  \item взаимодействие в web-сервисе: очереди, деление на компоненты, обращение к другим сервисам
  \item pubsub vs pubconsume, redis, pgq, rabbtimq, apache kafka и другие buzzwords на тему очередей...
  \item JavaScript - однопоточный, явная кооперативная многозадачность, ajax, timers, css3 animation, jquery.deffered()
  \item PHP - нет потоков*, "начинаем сначала" каждый раз, персистетность соеденения
  \item ruby (on rails) - огромное явление, редкие многопоточные применения, event machine, rack: middleware
  \item python - pure, gevent, twisted, ...
  \item Java - NIO, NIO2, etc ...
  \item Go (горутины), комбинированный вариант (M:N), неблокирующийся ввод-вывод, каналы
  \item Erlang - комбинированная многозадачность (Actor model), полная изоляция, распределённые процессы
  \item Андрей Аксёнов: цена абстракции
  \item личный странный опыт - жжот, всё остальное деньги-на-ветер
  \item vector<int> vs vector<int>+reverse vs int[12]+int*res+EOF === 0 vs +20\% vs +40\%
  \item на самом деле vector разумеется тормозит, а reverse(+=1000) очень быстр, попали в кэш?!...
  \item ай, доклады Аксёнова на бесполезно конспектировать, это надо видеть
  \item CRC32 - очень старый, медленный, и вообще не хеш-функция, рулят: FNV, Jenkins, MurMur, etc... теоретически, но не практически!
  \item А про SOA, на ничего не напишу - ноутбук сел...
  \item newsql от echo: comeback классических подходов
  \item sharding in app - дополнительная логика, больше серверов для поддержки, сложность поддержки
  \item архитектурное изменение: база данных уходить в облако, сервера становяться менее мощными и их количество возрастает
  \item P в CAP вообще-то не дискретно, границу каждый себе может провести сам; NoSQL базы обычно акцентируются на A
  \item NewSQL - DBMS с маштабируемостью и гибкостью, которая поддерживает SQL и/или поддерживает ACID транзакции
  \item shared nothing - каждая нода независима, бесконечное маштабирование, но любой запрос во все ноды будет очень медленным
  \item In-memory storage: быстрый доступ, высокая пропускная способность, никакого буфер-management-а, нет проблема с блокировками и тд
  \item newsql категории: новые решения: VoltDb, Clusterix, NuoDB; новый storage engine: TokuDB, scaleDB; прозрачная кластеризация: ...
  \item NewSQL это уже установивишийся тренд; но поскольку решения сильно различны, а универсального теста не существует ...
  \item Как устроен NoSQL, Андрей Аксёнов
  \item строки == документы == объекты; колонки == поля == значения == ..., CSV, SQL, XML, JSON, WTF... а суть одна - документы и их части
  \item структуры данных: как храняться данные (документы)? какой внутри формат строки? как храняться индексы?
  \item NoSQL "revolution", было: plain text, B-tree. Добавились: LSM, Bitcask, column-based. И алгоритмы компрессии: LZO, LZ4, snappy
  \item индексы: было B-tree, добавились: LSM+Bloom (псевдоиндекс по PK), Column Based... NB: вторичный - всегда btree(sic!)
  \item про мишуру, NoSQL "революция" - отсуствие схем, сплошной json, денормализация, шардинг, плюс мишура: rest/json синтаксис запросов
  \item Не забудь почитать в wikipedia про: B-Tree, LSM, Bloom
  \item а так же о Bitcask, column based, pfor
  \item Tarantool: redis + node.js для lua, все данные в одной процессе, кооперативная многозадачность, вторичные и multipart ключи
  \item tarantool 1.6 - база данных встроенная в интерпритатор lua - как drop-in replacement
  \item так или иначе, если убрать в сторону весь хайп вокруг json, всё боле популярным становится msgpack
  \item Tarantool + LuaJit = хранимые процедуры на языке общего назначения со скоростью C, интеграция с C => большое ускорение
  \item tarantool 1.6 - мастер-мастер асинхронная репликация(но синхронизация конфликтов на стороне app) - проще делать failover
  \item что ж, докатились и до биллинга в Badoo ...
  \item badoo - нужно понимать платежи со всего мира, монетизация - микроплатежи, импульсивные платежи, из-за объёма важен каждый \%
  \item в Badoo - отдел "билинга" - 15 человек, каждый может сравнить со своей компанией, я со своей сравнил - очень похоже :) ...
  \item билинг в badoo - отдельный сервис с API - HTTP/JSON для внешних и внутренних клиентов,  интеграция с игрегаторами...
  \item любой большой проект интегрируется с кучей партнёров - the must, ибо риски, ибо комиссия, разная реализация, ...
  \item совет для маленьких которые стремяться в большие: в каждой стране самый популярный метод оплаты - свой!
  \item MCC (Merchant Category Code) оказывается очень важная штука для повышения процента успешных платежей
  \item в сша после включения 3d secure платежи упали в 0. 
  \item от отключения 3d secure в рф число успешных платежей выросло с 70 до 85\% 
  \item работа с локальным экваером: +20\% сразу во многих странах
  \item у Badoo платежей с Android-ов больше чем с iOS, Google позволяет не использовать store, если сервис можно получить на сайте
  \item поддержка пользователей: наличие интерфейса для support-а где видно все платежи - позволяет решать 90\% support-а самостоятельно
  \item если делать билинг с нуля - нужно изначально делать как сервис, логировать!, мониторить!, следить за спамом и фродом
  \item Любой дополнительный шаг отнимает часть оплаты #платежи
  \item дальше - SOA от Wargaming, [реклама] мы стали это делать ещё до этого как подход сдох и снова стал быть модным :) ...
  \item vkontakte в Open Source
  \item KPHP; набор NoSQL движков: pmemcached, texts, lists, hints и т,д; набор утилит db-proxy, mc-proxy, antispam, cache
  \item KPHP - очень урезанный php, в 4 раза HHVM, в 25 раз PHP, +статический анализ кода; но -oop и отсуствие части стандартный lib
  \item NoSQL от VK - снимки и бинлоги, memcache протокол, rpc протокол, масштабируемость, скорость; -rdbms, только Linux
  \item pmemcached: KV + высокая скорость работы + набор ключей по префиксов + метафайлы
  \item lists - движок для списков, в качестве значения int и 256 символов текста
  \item Search - некоторая алтернатива sphinx: произвольные теги для выборки, произвольные поля для сортировки. Search-X и Search-Y - моды
  \item Storage - хранение медиа-контента, быстрее чем в отдельных файлах, работа через MC и RPC протоколы, администрирование+миграция
  \item hints - быстрый поиск по префиксам слов, поиск по объектам, сортировка объектов по рейтингу
  \item Queue - позволяет организовать realtime общения между браузером и сервером
  \item Friends - операции с приватностью, папки друзей, ряд специфических запросов
  \item letters - организация очереди для серверной обработки, события с произвольным количеством полей, повторы, удаления
  \item Logs - SQL движок предназначенный для логирования, хранит произвольные данные в течении определённого промежутка времени
  \item Bayes - предназначен для анализа текста на спам, необходимо обучать, возвращает вероятность того, что текст является spam-ом
  \item image engine, cache, CopyFast, Replicator - утилиты от VK для highload проектов
  \item VK до сих пор очень маленькая компания, 10 программистов (5 C и 5 PHP грубо) и 6 админов, даже не верится
